\documentclass[a4paper,12pt,DIV=calc]{scrartcl}

% Character encoding
\usepackage[utf8]{inputenc}
\usepackage[T1]{fontenc}

% Math + CSP
\usepackage{amsmath}
\usepackage{zed-csp}

\begin{document}

\title{PCOC Spr/Week 4 Formative}
\author{James Cowgill}
\maketitle

\begin{quote}
\itshape
Prove the validity in the traces model of the following law. Justify each step
of your proof by referring to the definition of the traces semantic function.

\[a \then (P \intchoice Q) = (a \then P) \intchoice (a \then Q)\]

Can FDR help you here?
\end{quote}

Under the traces model, we need to prove that this is always true:
\[
\Traces[a \then (P \intchoice Q)] =
\Traces[(a \then P) \intchoice (a \then Q)]
\]

The left hand side of the traces equation expands to the following using the
definition of $\Traces[a \then P]$.
\[
\Traces[a \then (P \intchoice Q)] \\ 
= \set{\nil} \union
\set{tr_p : \Traces[P \intchoice Q] \spot \trace{a} \cat tr_p}
\]

It can be furthur expanded using the definition of $\Traces[P \intchoice Q]$:
\[
= \set{\nil} \union
\set{tr_p : \Traces[P] \union \Traces[Q] \spot \trace{a} \cat tr_p}
\]

The set comprehension above iterates over the union of two sets ($\Traces[P]$
and $\Traces[Q]$). This can be rewritten as a union of two separate set
comprehensions.
\[
= \set{\nil} \union
\set{tr_p : \Traces[P] \spot \trace{a} \cat tr_p} \union
\set{tr_q : \Traces[Q] \spot \trace{a} \cat tr_q}
\]

We can then add a union with the empty sequence (which does nothing since we
already have it in our final of traces). Then we can use the two definitions
used previously to simplify and get the right hand side.
\[
= \set{\nil} \union
\set{tr_p : \Traces[P] \spot \trace{a} \cat tr_p} \union
\set{\nil} \union
\set{tr_q : \Traces[Q] \spot \trace{a} \cat tr_q} \\
= \Traces[a \then P] \union \Traces[a \then Q] \\
= \Traces[(a \then P) \intchoice (a \then Q)]
\]

FDR can't help here because it can't prove general laws - it can only prove
things about specific processes $P$ and $Q$.

\end{document}
