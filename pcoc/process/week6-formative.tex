\documentclass[a4paper,11pt,DIV=calc]{scrartcl}

% Character encoding
\usepackage[utf8]{inputenc}
\usepackage[T1]{fontenc}

% Math + CSP
\usepackage{amsmath}
\usepackage{zed-csp}

\begin{document}

\title{PCOC Spr/Week 6 Formative}
\author{James Cowgill}
\maketitle

\begin{quote}
\itshape
In the failures-divergences model, prove that:

\[P \intchoice (Q \extchoice R) = (P \intchoice Q) \extchoice (P \intchoice R)\]
\end{quote}

This means that we need to prove that both the divergences and failures-bottom
are equal.

\begin{align*}
  \Divergences[P \intchoice (Q \extchoice R)] &=
  \Divergences[(P \intchoice Q) \extchoice (P \intchoice R)] \\
  \Failures_\bot\semb{P \intchoice (Q \extchoice R)} &=
  \Failures_\bot\semb{(P \intchoice Q) \extchoice (P \intchoice R)}
\end{align*}

\section{Divergences}

It's fairly straightforward to show that the divergences of both sides are
equal.

\begin{align*}
  &\Divergences[P \intchoice (Q \extchoice R)] \\
  &= \Divergences[P] \cup \Divergences[Q \extchoice R] \\
  &= \Divergences[P] \cup \Divergences[Q] \cup \Divergences[R] \\
  &= \Divergences[P] \cup \Divergences[Q] \cup \Divergences[P] \cup
     \Divergences[R] \\
  &= \Divergences[P \intchoice Q] \cup \Divergences[P \intchoice R] \\
  &= \Divergences[(P \intchoice Q) \extchoice (P \intchoice R)] \\
\end{align*}

\section{Failures}

The $\Failures_\bot$ function is defined as:
\[
\Failures_\bot\semb{A} =
  \Failures[A] \cup \{s; ref | s \in \Divergences[P] \spot (s, ref)\}
\]

Since the divergences of both sides of the original equation are equal, the part
of the definition on the right hand side of the $\cup$ will also be equal so we
only need to show that the failures of both sides of the original equation are
equal to prove it.

Starting with the RHS:
\begin{align*}
  &\Failures[(P \intchoice Q) \extchoice (P \intchoice R)] \\
  &= \{tr, ref | \\
  &\phantom{===} tr = \nil \wedge (tr, ref) \in
    \Failures[P \intchoice Q] \cap \Failures[P \intchoice R] \\
  &\phantom{====} \vee \\
  &\phantom{===} tr \neq \nil \wedge (tr, ref) \in
    \Failures[P \intchoice Q] \cup \Failures[P \intchoice R] \\
  &\phantom{==}\spot (tr, ref)\} \\
  &= \{tr, ref | \\
  &\phantom{===} tr = \nil \wedge (tr, ref) \in
    (\Failures[P] \cup \Failures[Q]) \cap (\Failures[P] \cup \Failures[R]) \\
  &\phantom{====} \vee \\
  &\phantom{===} tr \neq \nil \wedge (tr, ref) \in
    (\Failures[P] \cup \Failures[Q]) \cup (\Failures[P] \cup \Failures[R]) \\
  &\phantom{==}\spot (tr, ref)\} \\
  &= \{tr, ref | \\
  &\phantom{===} tr = \nil \wedge (tr, ref) \in
    \Failures[P] \cup (\Failures[Q] \cap \Failures[R]) \\
  &\phantom{====} \vee \\
  &\phantom{===} tr \neq \nil \wedge (tr, ref) \in
    \Failures[P] \cup (\Failures[Q] \cup \Failures[R]) \\
  &\phantom{==}\spot (tr, ref)\}
\end{align*}

From the final line, it should be easy to see that if $(tr, ref) \in
\Failures[P]$, then one of the lines of the disjunction must be true. This means
we can extract $\Failures[P]$ into its own disjunction.

\begin{align*}
  &= \{tr, ref | \\
  &\phantom{===} (tr, ref) \in \Failures[P] \\
  &\phantom{====} \vee \\
  &\phantom{===} tr = \nil \wedge (tr, ref) \in
    \Failures[Q] \cap \Failures[R] \\
  &\phantom{====} \vee \\
  &\phantom{===} tr \neq \nil \wedge (tr, ref) \in
    \Failures[Q] \cup \Failures[R] \\
  &\phantom{==}\spot (tr, ref)\}
\end{align*}

The $\Failures[P]$ can now be split out of the set comprehension and the entire
thing can be simplified to the LHS of the original equation.

\begin{align*}
  &=\{tr, ref | \\
  &\phantom{===} tr = \nil \wedge (tr, ref) \in
    \Failures[Q] \cap \Failures[R] \\
  &\phantom{====} \vee \\
  &\phantom{===} tr \neq \nil \wedge (tr, ref) \in
    \Failures[Q] \cup \Failures[R] \\
  &\phantom{==}\spot (tr, ref)\} \cup \Failures[P] \\
  &= \Failures[P] \cup \Failures[Q \extchoice R] \\
  &= \Failures[P \intchoice (Q \extchoice R)]
\end{align*}

\end{document}
