\documentclass[a4paper,12pt,draft,DIV=calc]{scrartcl}

% Character Encoding
\usepackage[utf8]{inputenc}
\usepackage[T1]{fontenc}

% Math Typesetting
\usepackage{amsmath}
\usepackage{bussproofs}

\newcommand{\NullaryInfC}[1]{\AxiomC{}\UnaryInfC{#1}}
\newcommand{\NullaryProofTree}[1]{\begin{prooftree}\NullaryInfC{#1}\end{prooftree}}

% URLs
\usepackage{url}
\usepackage[pdfusetitle,hidelinks]{hyperref}

% Syntax Highlighting
\usepackage[final]{listingsutf8}
\usepackage{pxfonts}
\usepackage{color}

\definecolor{grey}{rgb}{0.3,0.3,0.3}
\lstset{
  commentstyle=\color{grey},
  basicstyle=\ttfamily
}

% Subfigures
\usepackage{subcaption}
\usepackage{verbatim}

% Various Typesetting Packages
\usepackage{microtype}
\usepackage[british]{babel}

\begin{document}
% Title
\title{EMBS WSN MAC Layer Protocol Report}
\author{James Cowgill}
\date{27th November 2015}
\maketitle

\section{Overview Stuff}
Report entry 1:
 - Design + Implementation Decisions
   - Synchronizing to multiple sinks
   - Handling of sinks with n = 1
 - References of the form (class name, method name, line numbers)
   - For every decision
 - Max 400 words, does not include references and figures
 - "well illustrated"

Report entry 2:
 - Design + Implementation Decisions
   - Synchonizing to multiple sinks
   - Design decisions related to energy efficiency
 - References as above
 - Max 500 words as above

\section{Exercise 1 - Ptolemy II}
- Mathematical part split into two parts
  - Data + operations related to a single Sink (SinkSyncData)
    - Calculation of n and delta t
    - Calculation of when to send the next packet (knowing n and delta t)
  - Multiple sink coordination (SourceController)
    - Owns one SinkSyncData object for each sink and demultiplexes events to
      them
    - Controls when to change channels
    - Controls when to wakeup (to send packets or for channel hopping)

- ptolemy II Model
  - Completely implemented in Java (no other ptolemy actors)
  - Wrapper around SourceController class
  - Sends events to the SourceController class
  - Obtains data and tried to set the "state of the world" to what
    SourceController wants

\section{Exercise 2 - Mote Runner}
- Code for mathemetical part is the same as the above
  - Why?
    - Code reuse - much less code to write!
    - Allows simulation in ptolemy of the ACTUAL code which will be used
    - Above code was designed to be fairly resiliant to packet loss
- Channel changing
  - Requires nothing being transmitted and the receiver must be stopped
    - Noticed some strange (undocumented) values returned by getState so I
      decided to ignore that method and use txOn and rxOn boolean variables to
      record it myself.
- SourceController does not have separate timers for things so actions are
  handled after events in a single "refresh" method handleControllerStateChange
  - Handles any previously pending transmit / receive events.
  - Attempts to transmit packets or start listening on a particular channel
  - If this failed (channel change), the event is deffered until both txOn
    and rxOn are false.

- Clock skew? (my code doesn't deal with it too much)

\end{document}
