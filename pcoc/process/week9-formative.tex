\documentclass[a4paper,11pt,DIV=calc]{scrartcl}

% Character encoding
\usepackage[utf8]{inputenc}
\usepackage[T1]{fontenc}

% Math + CSP
\usepackage{mathtools}
\usepackage{zed-csp}

% Custom macros
\newcommand{\Iter}{\mathit{Iter}}
\newcommand{\STOP}{\mathit{STOP}}
\newcommand{\SKIP}{\mathit{SKIP}}

\begin{document}

\title{PCOC Spr/Week 9 Formative}
\author{James Cowgill}
\maketitle

\begin{quote}
\itshape
Calculate the traces of the process
\[
  \Iter(c_1 \then \SKIP \extchoice c_1 \then c_2 \then \SKIP)
\]
\end{quote}

If $P$ is let to be the process being iterated, the traces of $P$ can be
calculated quite straightforwardly.

\begin{align*}
  \Traces[P]
  &= \Traces[c_1 \then \SKIP \extchoice c_1 \then c_2 \then \SKIP] \\
  &= \Traces[c_1 \then \SKIP] \cup \Traces[c_1 \then c_2 \then \SKIP] \\
  &= \{\nil, \trace{c_1}, \trace{c_1, \tick}\} \cup
     \{\nil, \trace{c_1}, \trace{c_1, c_2}, \trace{c_1, c_2, \tick}\} \\
  &= \{\nil, \trace{c_1}, \trace{c_1, \tick}, \trace{c_1, c_2}, \trace{c_1, c_2, \tick}\}
\end{align*}

The traces of the sequential composition $A \comp B$ consist of the traces of
$A$ which do not end with a $\tick$ event, in union with with the traces of $A$
which do end with a $\tick$ in sequence with any of the traces of $B$.
\begin{multline*}
  \Traces[A \comp B] =
    \Traces[A] \setminus \set{t \spot t \cat \trace{\tick}} \\ \cup
    \set{s, t \mid s \cat \trace{\tick} \in \Traces[A] \land t \in \Traces[B] \spot s \cat t}
\end{multline*}

The $\Iter$ process is defined as:
\[\Iter(Q) = Q \comp \Iter(Q)\]

The process $\Iter(P)$ (from the question) has the fixed point:
\[F(X) = P \comp X\]

The usual method of repeatedly applying the function $F$ to evaluate fixed
points can then be used to calculate the traces.

\begin{align*}
  \Traces[\STOP] =& \nil \\
  \Traces[F(\STOP)]
  =& \Traces[P \comp \STOP] \\
  =& \{\nil, \trace{c_1}, \trace{c_1, c_2}\} \cup \{\trace{c_1}, \trace{c_1, c_2}\} \\
  =& \{\nil, \trace{c_1}, \trace{c_1, c_2}\} \\
  \Traces[F(F(\STOP))]
  =& \Traces[P \comp F(X)] \\
  =& \{\nil, \trace{c_1}, \trace{c_1, c_2}\}\\
  & \cup \{\trace{c_1}, \trace{c_1, c_1}, \trace{c_1, c_1, c_2},
    \trace{c_1, c_2}, \trace{c_1, c_2, c_1}, \trace{c_1, c_2, c_1, c_2}\} \\
  =& \{\nil, \trace{c_1}, \trace{c_1, c_1}, \trace{c_1, c_1, c_2},
    \trace{c_1, c_2}, \trace{c_1, c_2, c_1}, \trace{c_1, c_2, c_1, c_2}\}
\end{align*}

The final set of traces will include any trace consisting of only the events
$c_1$ and $c_2$ such that any $c_2$ event immediately follows a $c_1$ event.

\end{document}
